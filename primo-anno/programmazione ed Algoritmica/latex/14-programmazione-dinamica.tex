% !TeX spellcheck = it_IT
\newpage
\section{Programmazione dinamica}
Si applica negli algoritmi ricorsivi in cui i sotto problemi ottenuti dalla tecnica \emph{Divide et impera} si ripropongono più volte, causando uno spreco di risorse considerevole. Si dice che questi sotto problemi non sono \textbf{indipendenti}. \\
La tecnica consiste nel memorizzare le soluzioni in una \textbf{tabella} dei sotto problemi in modo da potervi accedere quando li si incontra di nuovo senza doverli risolvere nuovamente.
\begin{example}[Fibonacci]
	Generare la sequenza di Fibonacci, che ricordiamo essere definita come:
	\begin{equation*}
		F_n = 
		\begin{cases}
			0 & n=0 \\
			1 & n=1 \\
			F_{n-1} + F_{n-2} & n \geq 2
		\end{cases}
	\end{equation*}
	\begin{lstlisting}[language=C, caption=Fibonacci, mathescape=true]
		Fib(n)
			if(n==0) return 0;
			if(n==1) return 1;
			return Fib(n-1)+Fib(n-2);
	\end{lstlisting}
	In questa soluzione di codice le somme eseguite sono in numero esponenziale:
	\begin{equation*}
		T(n) = 
		\begin{cases}
			0 & n \leq 1 \\
			1 + T(n-1) + T(n-2) & n > 1
		\end{cases}
	\end{equation*}
	Per calcolare la successione di Fibonacci con \textbf{memoization} di tipo \textbf{top down} usiamo il seguente algoritmo, sempre ricorsivo:
	 \begin{lstlisting}[language=C, caption=Fibonacci dinamico top-down, mathescape=true]
	 	Fib(n)
	 		F: new array(0:n)	// Dimensione n+1
	 		for i=0 to n
	 			F(i) = -1
	 		return FibRic(n, F)
	 	
	 	FibRic(n, F)
	 		if(n==0 || n==1) return n;
	 		if(F(n) != -1) return F[n]
	 		else
	 			F[n] = FibRic(n-1, F) + FibRic(n-2, F);
	 			return F[n];
	 \end{lstlisting}
 	Utilizzando invece il metodo \textbf{bottom-up}:
 	\begin{lstlisting}[language=C, caption=Fibonacci dinamico bottom-up, mathescape=true]
 		Fib(n)
 			F: new array(0:n)	// Dimensione n+1
 			F[0] = 0;
 			F[1] = 1;
	 		for i=2 to n
	 			F(i) = F[i-1] + F[i-2]
 			return F[n]
 	\end{lstlisting}
 	La complessità di questi algoritmi è $\Theta(n)$ in tempo e $\Theta(n)$ in spazio, a differenza dell'algoritmo non dinamico che ha una complessità di $\phi^n$ in tempo e %TODO Boh? so solo che non è lineare perché anche se non uso strutture dati di supporto ho lo stack dei record di attivazione dato che non è ricorsivo
 	Per ottimizzare l'algoritmo in spazio possiamo fare nel modo seguente:
 	\begin{lstlisting}[language=C, caption=Fibonacci spazio costante, mathescape=true]
 		Fib(n)
 			if(n==0) return 0;
 			if(n==1 || n==2) return 1;
 			a = 1
 			b = 1
 			for i=3 to n
 				c = a+b
 				a = b
 				b = c
 			return b;
 	\end{lstlisting}
 	\begin{observation}[Complessità in spazio]
 		Come sappiamo il numero di cifre necessarie per rappresentare un numero $n$ è $\log_x n$ dove $x$ è la base in cui scriviamo il numero. Di conseguenza in quest'ultimo algoritmo in realtà è \textbf{pseudo-polinomiale} in quanto passiamo da un'istanza di input logaritmica ad una complessità lineare.
 	\end{observation}
\end{example}

\subsection{Struttura di un algoritmo }
La programmazione dinamica si applica a problemi di ottimizzazione con queste caratteristiche:
\begin{itemize}
	\item \textbf{Sottostruttura ottima}: la soluzione ottima del problema si può costruire dalle soluzioni ottime dei sottoproblemi
	\item \textbf{Sovrapposizione dei sottoproblemi} (o ripetizione)
\end{itemize}
Gli algoritmi sono costituiti da 4 fasi:
\begin{enumerate}
	\item Definizione dei sotto problemi e dimensionamento della tabella.
	\item Soluzione diretta dei sotto problemi elementari e inserimento di questi nella tabella (approccio \textbf{bottom-up})
	\item Definizione della regola ricorsiva per ottenere la soluzione di un sotto problema a partire dalle soluzioni dei sotto problemi già risolti (già nella tabella)
	\item Restituzione del risultato relativo al problema di dimensione $n$
\end{enumerate}

\begin{example}[Taglio della corda]
	Data una corda di lunghezza $n$ e una tabella di prezzi (un pezzo di dimensione diversa ha un valore diverso), trovare il taglio ottimale della corda per massimizzare il guadagno.\\
	Un possibile modo di affrontare il problema è tramite \textbf{brute-force}, che comporta analizzare ogni possibile taglio della corda. Notiamo che possiamo dividere la corda in $2^{n -1}$ possibili modi.\\
	Un approccio \textbf{ricorsivo} al problema è quello di tagliare la corda al punto $i$. In questo modo abbiamo che il ricavo massimo ottenibile sarà il prezzo del pezzo tagliato $p(i)$ e il ricavo massimo della corda restante $r_{n-i}$. In generale il ricavo massimo quindi sarà $r_n = max_{1 \leq i \leq n}(p(i)+r(n-i))$.
	\begin{lstlisting}[language=C, caption=Taglio della corda, mathescape=true]
		CUT_ROD(P, n)
			if(n==0) return 0;
			q = -$\infty$
			for i=1 to n
				q = max{q, p[i] + CUT_ROD(P, n-i)}
			return q;
	\end{lstlisting}
	Questo algoritmo, per quanto breve, è estremamente inefficiente (esponenziale) a causa della ripetizione degli stessi sotto problemi. È quindi necessaria la \emph{programmazione dinamica}.
\end{example}

\begin{example}[Longest common subsequence]
	
\end{example}

\begin{example}[Edit distance]
	Date due parole $X$, $Y$ determinarne la distanza.\\
	La prima fase è quella dell'\textbf{allineamento}: ad ogni carattere o spazio di $X$ corrisponde un carattere o uno spazio di $Y$.\\
	Una volta eseguito l'allineamento, si calcola la distanza con queste regole:
	\begin{itemize}
		\item \textbf{MATCH} (caratteri corrispondenti uguali) $\rightarrow$ distanza + 0
		\item \textbf{MISMATCH} (caratteri corrispondenti diversi) $\rightarrow$ distanza + 1
		\item \textbf{SPACE} (carattere e spazio) $\rightarrow$ distanza + 1
	\end{itemize}
	Il problema della edit-distance è determinare la distanza \emph{minima}.
\end{example}

\subsection{Tecnica Greedy}
\begin{example}[Zaino frazionario]
	Il problema è quello dello zaino visto in precedenza ma in questo caso il ladro può prendere anche frazioni di oggetti.
	
	\textbf{Correttezza}: occorre dimostrare che il problema soddisfi gli elementi della tecnica greedy:
	\begin{itemize}
		\item \emph{Sottostruttura ottima}
		\item \emph{Proprietà della scelta greedy}: per assurdo, supponiamo che qualche scelta non sia greedy (localmente ottima). Anziché scegliere un oggetto $m$ di valore specifico $\frac{v_m}{w_m}$ , scegliamo $p$ kg dell'oggetto $q$ di valore specifico $\frac{v_q}{w_q}$.
		\begin{equation*}
			\frac{v_q}{w_q} < \frac{v_m}{w_m}
		\end{equation*}
		Sia $r=min\{p, w_m\}$
		%TODO Ti sei perso
	\end{itemize}
\end{example}

\begin{example}[Scheduling delle attività]
	Ogni attività inizia e finisce in due istanti di tempo diversi. \\
	\begin{table}[!h]
		\centering
		\begin{tabular}{|c|c|c|c|c|c|c|c|c|c|c|c|}
			\hline
			\textbf{i} & 1 & 2 & 3 & 4 & 5 & 6 & 7 & 8 & 9 & 10 & 11 \\
			\hline
			\textbf{$S_i$} \\
			\hline
			\textbf{$T_i$} \\
			\hline
		\end{tabular}
	\end{table}
	Il problema consiste nel massimizzare il numero di attività eseguibili rispettando il vincolo di \emph{non sovrapposizione}.\\
	Utilizziamo la strategia greedy:
	\begin{itemize}
		\item Seleziona l'attività che finisce prima
		\item Elimina le attività che non sono compatibili, ovvero che si sovrappongono con quella selezionata
		\item Ripeti
	\end{itemize}
\end{example}