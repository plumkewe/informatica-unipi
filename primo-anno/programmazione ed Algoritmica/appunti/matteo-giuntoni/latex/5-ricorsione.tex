\newpage
\section{Ricorsione}
\begin{definition}[Ricorsione]
	A tempo di \textbf{compilazione}: una funzione usa il suo nome (chiama se stessa) nel suo corpo.
	A tempo di \textbf{esecuzione}: chiamate annidate della \textbf{stessa} funzione
\end{definition}
\noindent Una funzione ricorsiva è chiamata per risolvere un problema scomposto in:
\begin{itemize}
	\item \textbf{Caso base}: la funzione restituisce un valore
	\item \textbf{Passo ricorsivo}: la funzione viene chiamata su un problema analogo a quello iniziale ma di dimensioni minori, avvicinandosi al \emph{caso base}
\end{itemize}
Quando si  arriva al caso base viene effettuata una sequenza inversa di return statement, combinando i risultati parziali in quello finale.
\begin{example}[Fattoriale]
	Il fattoriale di un intero non negativo n è il prodotto
	degli interi positivi $<= n$ escluso lo $0$. Si indica con
	$n!$ e si impone per definizione $0! = 1$.
	\begin{equation}
		n! = \prod_{i=1}^{n} i=n*(n-1)*\ldots*1
	\end{equation}
	oppure definita in maniera ricorsiva:
	\begin{equation}
		n! = \begin{cases}
			1, \hspace{50px} n=0 \\
			n*(n-1)!, \hspace{6px} n>0
		\end{cases}
	\end{equation}
	In maniera programmatica possiamo scriverlo come:
	\begin{lstlisting}[language=Swift, caption=Fattoriale con ricorsione, mathescape=true]
		func F(var n: Int) -> Int {
			if (n-1) {
				return 1
			} else {
				return n * F(n-1)
			}
		}
	\end{lstlisting}
\end{example}

\subsection{Ricorsione e iterazione}
\begin{tabular} { |c|p{150px}|p{150px}|}
	\hline
	& \textbf{Ricorsione} & \textbf{Iterazione} \\
	\hline
	Controllo di terminazione & Condizione di ricorsione & Condizione di controllo nel loop \\
	\hline
	Ripetizioni & Chiamate ricorsive della funzione & Esecuzione ripetuta del corpo dell'iterazione \\
	\hline
	Convergenza alla terminazione & I passi ricorsivi riducono il problema al caso base & Il contatore si avvicina al valore di termine \\
	\hline
	Ripetizione infinita & Il passo ricorsivo non riduce il problema e non si avvicina al caso base & La condizione di controllo non è mai falsa \\
	\hline
\end{tabular}
\vspace{15pt}

\noindent Nella \emph{ricorsione}, al contrario dell'\emph{iterazione}, ogni chiamata alla funzione genera un nuovo record di attivazione contenente una nuova copia delle variabili e consumando lo stack di esecuzione. Questo può generare \textbf{overhead}.\\
In generale ogni problema \emph{ricorsivo} può essere anche scritto \emph{iterativamente}. È consigliato scriverlo ricorsivamente quando ciò facilita la lettura del problema stesso.