\newpage
\section{Condizioni}
\subsection{Condizioni su array}
Dato un array \textbf{a} di dimensione N, voglio verificare se la proprietà P vale per tutti gli elementi dell'array.
\begin{equation}
	\forall i \in [0, N).\mathcal{P}(a[i])
\end{equation}
\begin{example}
	Verifico che tutti gli elementi dell'array siano dispari.
	\begin{equation}
		\forall i \in [0, N).a[i]%2==1
	\end{equation}
	\begin{lstlisting}[language=C, caption=Verifica di proprietà su tutti gli elementi mathescape=true]
		int check_array_dispari(int a[], size_t dim) {
			int indice = 0;
			while (indice < dim && a[indice]%2 == 1){
				indice++;
			}
			if (indice == dim) {
				return 1;
			} else {
				return 0;
			}	
		}
	\end{lstlisting}
	Blocco lo scorrimento dell'array quando la proprietà \textbf{NON} viene soddisfatta almeno una volta.
\end{example}

\noindent
Se invece voglio verificare che la proprietà P valga per almeno un elemento:
\begin{equation}
	\exists i \in [0, N).\mathcal{P}(a[i])
\end{equation}
\begin{example}
	Verifico che almeno un elemento dell'array è uguale a 26.
	\begin{equation}
		\exists i \in [0, N).a[i]==26
	\end{equation}
	\begin{lstlisting}[language=C, caption=Verifica di proprietà su almeno un elemento, mathescape=true]
		int esiste_in_array(int a[], size_t dim, in n) {
			size_t indice = 0;
			_Bool trovato = 0;
			while (indice < dim && !trovato){
				if(a[indice] == n) {
					trovato = 1;	
				}
				indice++;
			}
			return trovato;
		}
	\end{lstlisting}
	Blocco lo scorrimento dell'array nel momento in cui trovo un elemento che soddisfa la proprietà, utilizzando un \textit{flag}.
\end{example}
\subsection{Condizioni su matrici}
Una \textbf{matrice} è un array di array. Può essere \textit{multidimensionale} $N \times M$ e voglio verificare se tutti i suoi elementi oppure solo uno di essi verificano una proprietà P.
\begin{equation}
	\forall i \in [0, N), \forall j \in [0,M).\mathcal{P}(a[i,j])
\end{equation}
\begin{equation}
	\exists i \in [0, N), \exists j \in [0,M).\mathcal{P}(a[i,j])
\end{equation}
\begin{definition}[Matrice quadrata]
	Una matrice è \textbf{quadrata} se a lo stesso numero di righe e di colonne. In questo caso per scorrerla si può usare un solo indice:
	\begin{equation}
		\exists i,j \in [0, N).\mathcal{P}(a[i,j])
	\end{equation}
\end{definition}

\begin{example}
	Verifico se tutti gli elementi della matrice sono positivi.
	\begin{equation}
		\forall i \in [0, N), \forall j \in [0, M) . a[i, j] > 0
	\end{equation}
	\begin{lstlisting}[language=C, caption=Verifica di proprietà su tutti gli elementi della matrice, mathescape=true]
		int check_matrice_pos(int a[][COL], size_t dim) {
			size_t row, col;
			row = col = 0;
			while (row < dim && a[row][col] > 0) {
				col = 0;
				while (col < COL && a[row][col] > 0) {
					col++;
				}
				if (col == COL) {
					row++;
				}
			}
			if (row == dim && col == COL) {
				return 1;
			}
			else {
				return 0;
			}
		}
	\end{lstlisting}
\end{example}
\begin{definition}[Matrice simmetrica]
	Una matrice è \textbf{simmetrica} se è quadrata e se le posizioni simmetriche rispetto alla diagonale principale contengono gli stessi elementi.
\end{definition}
\begin{definition}[Matrice triangolare]
	Una matrice è \textbf{triangolare} superiore o inferiore se le posizioni rispettivamente sopra o sotto la diagonale contengono tutti 0.
\end{definition}
\begin{definition}[Matrice tridiagonale]
	Una matrice \textbf{tridiagonale} \color{red} può \color{black} avere elementi non nulli solo sulla diagonale principale e la sua diagonale superiore ed inferiore.
\end{definition}

\subsection{Contare elementi che verificano una proprietà}
Dato un array \textbf{a} di dimensione $N$ per contare tutti gli elementi che verificano una proprietà $P$:
\begin{equation}
	\#\{i \lvert i\in [0, N-1] \land \mathcal{P}(a[i])\}
\end{equation}
Data invece una matrice \textbf{a} di dimensione $N \times M$:
\begin{equation}
	\#\{(i,j) \lvert i \in [0, N-1] \land j \in [0, M-1]\}
\end{equation}